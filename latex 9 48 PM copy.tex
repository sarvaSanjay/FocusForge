\documentclass[fontsize=11pt]{article}
\usepackage{amsmath}
\usepackage[utf8]{inputenc}
\usepackage{url}
\usepackage[margin=0.75in]{geometry}

\title {FocusForge - Creating a timetable to get the focus of your choice at UofT}
\author{Sarva Sanjay, Raahil Vora, Ansh Prasad}
\date{Sunday, April 2, 2023}

\begin{document}
\maketitle

\section*{Problem Description and Research Question}

At the University of Toronto, to choose a focus, you are required to take a series of required courses and have a certain number of credits to be considered for admission in your desired focus. In computer science, we have Courseography (Department of Computer Science, University of Toronto, 2013) to understand better how to achieve a focus. However, other faculties still need to implement this officially. Furthermore, Courseography does not provide when to take each required course and which semester. This information would be crucial for students to prepare and effectively select their courses for each semester without overburdening themselves while still obtaining the same outcome.
\\
\\Understanding which subjects to take and when could be a challenge as a student may need to know which subjects to take, have specific exclusions from courses, or even want to structure their future years in a balanced, efficient way. Alternatively, students may know what subjects they are passionate about but need clarification on which focus to take. 
\\
\\We were inspired by a discussion amongst our team about the required course to do different focuses in Computer Science. This discussion led to the goal to \textbf{create a desktop application to aid students in selecting courses at UTSG for future terms based on their completed courses and breadth requirements. Using the information, we will create a guide for when to take additional classes during their remaining years at university to be eligible to apply to a desired focus.} 
\\
\\The application will provide a better view of your degree by finding all course paths the user can take to reach a focus. It will help determine the shortest and most optimum path (and when to take it during their remaining time at UofT) so the user can comfortably select courses at UofT, knowing that it will benefit their degree. Furthermore, as mentioned previously, the program will also help students choose a focus based on the courses they have already completed and the courses they would like to take. This application would greatly ease students’ course and focus selection to have a smoother and less stressful experience with course decisions at UofT.

\section*{Datasets}
Our project includes two datasets: a dataset containing courses at UofT and a dataset containing the Focuses at UofT.\\
The dataset containing the courses was generated by scraping the UofT Faculty of Arts and Science's 2022-2023 Academic Calendar and storing them in the csv file course-data.csv . The csv file contains three columns: the course's code (e.g., CSC265H1), the course's name (e.g.,Enriched Data Structures and Analysis), and the prerequisites for that course (e.g., CSC236 or CSC240)
\\ The dataset containing the focuses was first scraped (from \url{https://artsci.calendar.utoronto.ca/print/view/pdf/search_programs/print_page/debug?combine=&field_subject_area_prog_search_value=All&type=4} ). After this, we manually fixed errors in the focuses due to variable placements on the page, and stores them in a csv file with two columns: the focus's name, and the focus's requirements.


\section*{Computational Plan}
Our project focuses on the course and focus data from the University of Toronto website and represents our course data as a directed graph. We have extracted course information data by scraping the University of Toronto Arts and Science Calendar websites (2022-23 Calendar $\vert$ Academic Calendar, 2022) using the Beautiful Soup library and cleaned it to remove any unwanted data. Our next step was to represent this data as a directed graph. In our graph, we consider the courses to be analogous to vertices in the graph, and prerequisite relations between courses are described as edges. \\
In our Course class, we store the course code, the course name, it's credit worth, and its corresponding set of prerequisite courses. In addition, we also have a get\_prereqs method that returns a path from courses with no prerequisites to the given course by recursively calling itself for its prerequisite courses. Two notable differences exist between our implementation of a graph and the one in the course notes. First, since prerequisite relations are directional, edges are represented as tuples instead of sets. Second, since there are multiple ways to meet the prerequisites for a particular course, we must represent them as a list of sets instead of a single set.
\\ In addition to our graph and course class, we have also created a Focus class. For each focus, we store its name, focus code, total credit requirement and required courses. In addition, we also have two methods: credits\_left (which returns the number or credits remaining to complete a focus, given a set of completed courses) and get\_paths (which returns a list of set of courses where each set excludes courses from a given set of completed courses). 
\\ With this, our program starts by building the graph. It reads the 'courses-data.csv' csv file and then building a graph with no connections. After this, it reads the file again and adds prerequisites for each course. For GUI performance reasons, we have split the focus setup into two steps. A "minimal" focus, which only contains the name and credits required for a focus, and a "complete" focus, which takes in a minimal focus and adds the focus's required courses.
We then move to the UI\\
\begin{huge} Ansh here?
also sarva for schedule.
\end{huge}

% \begin{itemize}
 %   \item 
  %  \item A \texttt{recursive\_prerequisite\_finder} that calculates all possible paths in which one could complete the prerequisite for a particular course, from beginning to end.
   % \item A \texttt{calculate\_credits} method that, given a path, calculates the number of additional credits to be completed in addition to the ones already completed by the user. This method serves as a way to rank the paths and filter them since the path with the least credits is the one which is best for the user.
    %\item A \texttt{generate\_schedule} algorithm which maps courses in a path to the semester in which the user shall be completing them. For example, the path CSC110-$>$ CSC111-$>$ CSC207 shall give an output \{'Fall 2022': 'CSC110', 'Winter 2023': 'CSC111', 'Fall 2023': 'CSC207'\}.

    
%\end{itemize}

%The results of these algorithms shall be displayed in a desktop app using the Tkinter library in python. The user shall be able to select the focus of his choice and enter the courses he has already taken in the past, and the GUI shall display the various schedules he could have in the future. The GUI shall help the user visualize the way they shall complete their focus and their 'degree' in a better way.
\section*{Obtaining Dataset and Running}
extract datasets.zip into the root directory of FocusForge and install the  BeautifulSoup 4 library\\
\begin{huge}
All other requirements should be automatically satisfied. We can test later checking the repo on a new venv
\end{huge}
% Can't think of anything else to add
\section*{Changes}
After comparing both scrapy and beautifulsoup, we decided to go with beautiful soup. In addition, in contrast to our original plan, we were not able to create a scraper that could validly scrape the focus data, so we had to manually fix it.
\begin{huge} Can't think of anything else \end{huge}
\section*{Discussion}
Our program satisfies our goal. We are able to help students by proving the courses they have left to complete a focus and when to take them. However, we did have some issues. We had to manually fix our focus dataset (i.e., we can't automate it, so it can't work for any arbitrary number of focuses). Moreover, we also struggled to complete the schedule function. One of the biggest issues, however, is that our program is a bit slow, which might have been able to fix by using a custom datatype instead for our pathways to completion (for courses and focuses)


\section*{References}
\begin{enumerate}
    \item Department of Computer Science, University of Toronto. (2013). Courseography - Graph. Courseography. https://courseography.cdf.toronto.edu/graph
    \item 2022-23 Calendar $\vert$ Academic Calendar. (2022). https://artsci.calendar.utoronto.ca/
\end{enumerate}
    



% NOTE: LaTeX does have a built-in way of generating references automatically,
% but it's a bit tricky to use so we STRONGLY recommend writing your references
% manually, using a standard academic format like APA or MLA.
% (E.g., https://owl.purdue.edu/owl/research_and_citation/apa_style/apa_formatting_and_style_guide/general_format.html)

\end{document}